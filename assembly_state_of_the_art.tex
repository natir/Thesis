\documentclass[main.tex]{subfiles}

\begin{document}

\chapter{Assembly tools state of the art}

\section{General overview}

\OLC for Overlap Layout Consensus was the main paradigm used to assembly Sanger reads.
Two reads overlap when they share a common part at their ends.
Tools build a graph where reads are node and an edge was build when two reads overlap, this graph is the layout. The assembly is built by achieving the consensus of reads sharing a common region. The advantage of the OLC approach over the Greedy approach is the graph. Indeed a repetition larger than a reads create a cycle in the graph, this property avoid some misassembly.

\missingfigure{fig to present graph and effect of repetition in graph}

For third generation assembly OLC has been mostly privileged, with two main trouble, how to find overlap between nosy reads, and how to build a good quality consensus between this type of read. In this chapter we present how actual tools manage this trouble

\section{State of the art}

\subsection{Pipeline with correction}

\section{\canu}

\canu is based on \toolsname{Celera}\pim{add citation}, we can split the \canu pipeline in three step:
\begin{itemize}
    \item correction
    \item trimming
    \item assembly
\end{itemize}

to find overlap between long read is use \mhap based on skeetch. After

\subsection{Falcon}

\subsection{Pipeline without correction}

\section{Miniasm}

\section{HINGE}

\section{Ra}

\section{wgtdb2}

\onlyinsubfile{
\bibliographystyle{plainnat}
\bibliography{main}
\addcontentsline{toc}{chapter}{Bibliography}
}

\end{document}