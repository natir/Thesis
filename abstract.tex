\documentclass[main.tex]{subfiles}

\begin{document}

\pagestyle{empty}

\section*{Abstract}

The sequencing of genetic information provides better understanding for a large number of biological phenomena: e.g. genetic diseases, speciation events, fundamental mechanisms of cell function. Sequencing techniques have considerably evolved since the Sanger method (1977). Nowadays third-generation sequencing technologies greatly reduce the costs of sequencing complete genomes. They produce longer reads (sequence fragments), but require the design of specific assembly tools that take into account the high error rates in the produced fragments.

The study of methods used by third-generation read assembly pipelines has revealed that improvements in assembly were possible without modifying assembly tools themselves. Some improvements are thus proposed in this thesis work, and were implemented through publicly available tools. \yacrd and \fpa pre-process the set of reads prior to assembly, in order to improve efficiency and quality of the assembly process. \knot combines information from both the input reads and an assembly, in order to provide insights on how to improve the contiguity of an assembly.

\textbf{Keywords:} Genome assembly, Third generation sequencing, Assembly graphs

\section*{Résumé}

Le séquençage de l'information génétique a permis de mieux comprendre un grande nombre de phénomènes biologiques, maladies génétiques, évènements de spéciations, mécanismes fondamentaux du fonctionnement de nos cellules. Les techniques de séquençage ont beaucoup évolué depuis la méthode de Sanger (1977). De nos jours, les technologies de séquençage de troisième génération permettent le séquençage d'un génome complet à moindre coût, produisent des lectures (fragments de genomes) plus longs, mais nécessitent la création d'outils d'assemblage spécifiques pour tenir compte d'un taux d'erreur élevé dans les lectures produites.

L'étude des méthodes utilisées par les outils d'assemblage de lectures de troisième génération a permis d'observer que des améliorations des assemblages étaient possibles sans toutefois modifier les outils eux-mêmes. Certaines améliorations sont proposées dans ce travail de thèse, et sont mises en œuvre à travers des outils proposés à la communauté. \yacrd et \fpa  interviennent en amont de l'assemblage en lui-même pour améliorer l'ensemble des lectures données en entrée à un assembleur. \knot analyse et combine le résultat d'un assemblage avec les données brutes, pour donner des pistes permettant d'améliorer l'assemblage final.

\textbf{Mots clef:} Assemblage de génomes, Troisième generation de séquençage, Graphes d'assemblage

\end{document}