\documentclass[main.tex]{subfiles}

\begin{document}

\pagestyle{empty}

\section*{Abstract}

The sequencing of genetic information has allowed us to better understand a large number of biological phenomena, genetic disease, speciation events, fundamental mechanisms of cell function. Sequencing techniques have evolved considerably since the Sanger method in 1977, the third generation sequencinq has greatly reduced the costs of obtaining a complete genome due to their long length, but has also required the creation of new assembly tools to take into account their high error rates.

The study of the methods used by these new third-generation read assembly tools revealed that improvements in assembly were possible without modifying the assembly tools themselves. These improvements were implemented through the tools \yacrd and \fpa that occur before assembly to improve the input data of the assembly tool and \knot that by combining the input data of the assembly tool and the output data provided by the same assembly tool to improve the final assembly.

\textbf{Mots clef:} Genome assembly, Third generation sequencing, Graphes

\section*{Résumé}

Le séquençage de l'information génétique nous a permis de mieux comprendre une grande nombre de phénomènes biologique, maladie génétique, événement de spéciations, mécanisme fondamentaux du fonctionnement de nos cellules. Les techniques de séquençage on beaucoup évoluver depuis la méthode de Sanger en 1977, la troisième génération de séquenceur a permis de réduire grandement les couts d'obtention d'un génome complet gr ace a leurs grandes longueur, mais a aussi nécessité la création de nouveau outils d'assemblage pour tenir compte de leurs taux d'erreur élevés.

L'étude des méthodes utilisé par ces nouveaux outils d'assemblage de reads de troisième génération, a permis d'observer que des améliorations des assemblages été possible sans toutes fois modifier les outils d'assemblage eux même. Ces améliorations on étai mise en place à travers les outils \yacrd et \fpa qui intervienne avant l'assemblage pour améliorer les données d'entré des assembleurs et \knot qui en combinant les données d'entre de l'assembleur et celle fournis en sortie par ce même assembleur pour améliorais l'assemblage final.

\textbf{Mots clef:} Assemblage de génome, Troisième generation de séquençage, Graphes

\end{document}