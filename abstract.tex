\documentclass[main.tex]{subfiles}

\begin{document}

\pagestyle{empty}

\section*{Abstract}

The sequencing of genetic information allow a better understanding of a large number of biological phenomena, genetic diseases, speciation events, fundamental mechanisms of cell functioning. Sequencing techniques have considerably evolved since the Sanger method in 1977. Nowadays third generation sequencing technologies greatly reduce the costs of sequencing a complete genome, produce longer reads (sequence fragments), but require the design of specific assembly tools to take into account the high error rates of produced fragments.

The study of the methods used by third-generation read assembly pipelines revealed that improvements in assembly were possible without modifying assembly tools themselves. Some improvements are thus proposed in this work and were implemented through tools publicly available. \yacrd and \fpa pre-process the set of reads before starting the assembly by itself. \knot post-processes input reads and data structures provided by the assembly tool to improve the final assembly.

\textbf{Keywords:} Genome assembly, Third generation sequencing, Graphs

\section*{Résumé}

Le séquençage de l'information génétique a permis de mieux comprendre un grande nombre de phénomènes biologiques, maladies génétiques, évènements de spéciations, mécanismes fondamentaux du fonctionnement de nos cellules. Les techniques de séquençage ont beaucoup évoluver depuis la méthode de Sanger en 1977. De nos jours, les technologies de séquençage de troisième génération permettent le séquençage d'un génome complet à moindre coût, produisent des lectures (fragments de séquence) plus longs, mais nécessitent la création d'outils d'assemblage spécifiques pour tenir compte du taux d'erreur élevé des lectures produites.

L'étude des méthodes utilisées par les outils d'assemblage de lectures de troisième génération a permis d'observer que des améliorations des assemblages étaient possibles sans toutefois modifier les outils eux-mêmes. Certaines améliorations sont proposées dans ce travail et ont été mise en œuvre à travers des outils proposés à la communauté. \yacrd et \fpa  intervienne en amont de l'assemblage en lui-même pour améliorer l'ensemble des lectures données en entrée à un assembleur. \knot analyse et combine les données d'entrée de l'assembleur et les structures de données produites par l'assembleur pour améliorer l'assemblage final.

\textbf{Mots clef:} Assemblage de génome, Troisième generation de séquençage, Graphes

\end{document}