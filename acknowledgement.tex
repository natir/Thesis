\documentclass[main.tex]{subfiles}

\begin{document}

\section*{Remerciment}
\markboth{\MakeUppercase Remerciment}{\MakeUppercase Remerciment}%

Ces remerciments contrairement a cette thèse seront écris en Français et ne seront pas relue. En effet je suis dyslexique dysorthographique, je souhaite que ces remerciments ne soit pas corrigé pour montré le travail qui a été accomplis par toutes les personnes qui mon accompagné durant toutes ces années. Qui on relu mes lettres de motivation, rapport de stage, CV, la moindre de mes communications officiel. Donc je tiens a remercié ma mère qui c'est en tout premier chargé de ce travail, et mes camarades étudiant·e·s. Mais aussi les relecteurs de cette thèse, Antoine Limasette, Camille Marchet, …, Rayan Chikhi et Jean-Stéphane Varré.

Rayan Chikhi et Jean-Stéphane Varré qui on été mes directeurs de thèse je les remercies pour leurs aides durant ces trois années. Ils mon guidé dans et assisté pour mon entré dans le monde de la recherche.

Je souhaiterais aussi remercie les personnes qui mon supporté physiquement pendant ces 3 années je parle évidement de mes co-bureau. Tous d'abord Samuel, qui ma accueilli dans son bureau je me souviendrais toujours de nos discussions aussi bien scientifique que politique. Ensuite Antoine qui ma souvent aidée a réfléchire, enfin Mael et Arnaud qui mon supporté moins long temps certe mais durant la dernière phase de rédaction de ma thèse.

Je tiens aussi a remercie l'ensemble des membres de l'équipe Bonsai, du Laboratoire CRISTAL et de l'université de Lille 


Communauté bio-info français Bioinfo-fr jebif twitter

Ma famille

Je dédis cette thèse a Hélene ma tante et Jean-Paul mon grand père qui sont mort durant cette thèse.

\end{document}