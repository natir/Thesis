\documentclass[main.tex]{subfiles}

\begin{document}

\section*{Remerciment}
\markboth{\MakeUppercase Remerciment}{\MakeUppercase Remerciment}%

Ces remerciments contrairement a cette thèse seront écris en Français et ne seront pas relue même part moi. En effet je suis dyslexique dysorthographique, je souhaite que ces remerciments ne soit pas corrigé pour montré le travail qui a été accomplis par toutes les personnes qui mon accompagné durant toutes ces années. Qui on relu mes lettres de motivation, rapport de stage, CV, la moindre de mes communications officiel. Donc je tiens a remercié ma mère qui c'est en tout premier chargé de ce travail titanesque, et mes camarades étudiant·e·s. Mais aussi les relecteurs de touts ou partie de cette thèse, Antoine Limasette, Camille Marchet, Nicolas Vinarnick, Yoann Dufresne, Matthieu Falce, Kevin Gueuti, Pierre Morisse, Maxime Garcia, Guillaume Devailly, Jérome Pivert, Aurélien Beliard, bjonnh, Sacha Schutz, Rayan Chikhi et Jean-Stéphane Varré.

Je souhaiterais remercié \pim{todo} d'avoir accepté de faire partie de mon jury.

Je souhaite remercié aussi Rayan Chikhi et Jean-Stéphane Varré qui on été mes directeurs de thèse, pour leurs aides durant ces trois années. Ils mon guidé et assisté dans mon entré dans le monde de la recherche.

Je pense aussi aux personnes qui mon supporté physiquement pendant ces 3 années je parle évidement de mes co-bureau. Tous d'abord Samuel, qui ma accueilli dans son bureau je me souviendrais toujours de nos discussions aussi bien scientifique que politique. Ensuite Antoine qui ma souvent aidée a réfléchire, enfin Mael et Arnaud qui mon supporté moins long temps certe mais durant la dernière phase de rédaction de ma thèse.

Je tiens aussi a remercie l'ensemble des membres de l'équipe Bonsai, du Laboratoire CRISTAL et de l'université de Lille et du centre INRIA Lille Nord Europe qui mon accueuil durant cette thèse.

Je souhaiterais aussi remercié le blog Bioinfo-fr et l'association des Jeunes Bioinformaticiens Français (JEBIF) a travers ces organisation j'aimerais remercié  l'ensemble de la communauté bioinformatique Française pour son accueil bienviellant.

Je souhaiterais aussi remercie tous les amis que j'ai rencontré a Lille, les camarades du Syndicat Solidaires Étudiant·e·s, les biodouilleurs d'ABL, les libriste de Chtinux. Je prèfere ne cité personne nommément, pour être sur de ne froissé personne. Mais sachez que si vous vous reconnaissez dans un de ces groupes je vous doit et cette thèse vous doit beaucoup. Évidement j'ai une pensé particulière pour tous les individus qui on subis plus ou moins volontairement le compte twitter de ma thèse. 

Je ne peut rédiger ces remerciments sans cité, Léa, Zoé et Flavien, qui plus que d'autre mon aider a parfois oublié la thèse et qui l'on aussi un peut porté avec moi.

Je tiens a remercie évidement toutes ma famille pour leurs soutient et leurs assistance de tous les instants. Mais aussi mes nombreux coloc, et particulièrement Matthieux qui a fais attention a ce que je manger pendant la rédaction de cette thèse.


Je dédis cette thèse a Hélene ma tante et Jean-Paul mon grand père qui sont mort durant cette thèse.

\end{document}